\item \cppinline|string strn(T v)| \itt{strn}

\cppinline|string strn_sci(T v)| \itt{strn_sci}

These functions will convert the value \cppinline{v} into a string. This value can be of any type, as long as it is convertible to string. In particular, \cppinline{v} can be a vector, in which case the output string will contain all the values of the vector, separated by commas, and enclosed inside curly braces \cppinline{"{...}"}.

\begin{example}
\begin{cppcode}
strn(2);            // "2"
strn(true);         // "1"
strn("foo");        // "foo"
strn(vec1i{2,5,9}); // "{2, 5, 9}"
\end{cppcode}
\end{example}

The second version is dedicated to floating point numbers, and will output them in scientific format.

\begin{example}
\begin{cppcode}
strn_sci(2.0);    // "2.000000e+00"
strn_sci(2e10);   // "2.000000e+10"
strn_sci(-5e-2);  // "-5.000000e-02"

// Integer numbers are not affected
strn_sci(1000);   // "1000"
strn_sci(1000.0); // "1.000000e+03"
\end{cppcode}
\end{example}

\item \cppinline|vec_<D,string> strna(vec<D,T>)| \itt{strna}

\cppinline|vec_<D,string> strna_sci(vec<D,T> v)| \itt{strna_sci}

These functions are the vectorized version of \cppinline{strn} (see above). The reason why the name of the function is different is because it is already possible to call \cppinline{strn} with vector arguments: the whole vector will be converted into a \emph{single} string. These versions however convert each element separately to form a vector of strings.

\begin{example}
\begin{cppcode}
strna(vec1i{1,5,6,9}); // {"1", "5", "6", "9"}
strn(vec1i{1,5,6,9});  // "{1, 5, 6, 9}"
\end{cppcode}
\end{example}

\item \vectorfunc \cppinline|bool from_string(string s, T& v)| \itt{from_string}

This function tries to convert the string \cppinline{s} into a C++ value \cppinline{v} and returns \cpptrue in case of success. If the string cannot be converted into this value, for example if the string contains letters and the value has an arithmetic type, or if the number inside the string is too big to fit inside the C++ value, the function will return \cppfalse. In this case, the value of \cppinline{v} is not modified.

The vectorized version of this function will try to convert every value inside the string vector \cppinline{s}, and will store the converted values inside the vector \cppinline{v} (it will take care or properly resizing the vector, so you can pass an empty vector if you want). The return value will then be an array of boolean values, corresponding to the success or failure of each individual value inside \cppinline{s}. If an element of \cppinline{s} fails to convert, the corresponding value in \cppinline{v} will be default-initialized (e.g., zero for numbers).

\begin{example}
\begin{cppcode}
float f;
bool b = from_string("3.1415", f);
b; // true
f; // 3.1415

b = from_string("abcdef", f);
b; // false;
f; // 3.1415, 'f' has not been modified

vec1f fs;
vec1b bs = from_string({"1", "15", "abc", "1e128", "2.5"}, fs);
bs; // {true, true, false, false, true}
fs; // {1,    15,   0,     0,     2.5}
\end{cppcode}
\end{example}
