\funcitem \cppinline|void print(...)| \itt{print}

\cppinline|void error(...)| \itt{error}

\cppinline|void warning(...)| \itt{warning}

\cppinline|void note(...)| \itt{note}

These functions will print to the standard output the content of each of their argument, side by side without any spacing or separator, and end the current line. The only difference between the various functions listed above is that \cppinline{error()}, \cppinline{warning()} and \cppinline{note()} will append a prefix to the message, respectively \cppinline{"error: "}, \cppinline{"warning: "} and \cppinline{"note: "}. In some systems, these particular messages (including the prefix and the print arguments) may be colored to make them stand out better from the regular \cppinline{print()} output.

\cppinline{print()} is rather used for debugging purposes:
\begin{cppcode}
uint_t a = 5;
vec1u x = {1,2,3,4,9,10};
// Most likely, this line of code below will disappear once we are
// sure the program is working properly
print("the program reached this location, a=", a, ", and x=", x);
\end{cppcode}

The above code will simply display:
\begin{bashcode}
the program reached this location, a=5, and x={1,2,3,4,9,10}
\end{bashcode}

The other functions are much more important, as they are meant to talk to the \emph{user} of a program, for example to tell him/her that something went wrong. If this is a serious error and the program has to stop, use \cppinline{error()}. This could indicate an error from the user of the program (i.e., wrong input), or from the person who wrote the code (i.e., ``the program should never go there, I'll just make it an explicit error just in case''. This is good practice!). Else if this is a recoverable error or simply a potential problem (maybe important, maybe not), use \cppinline{warning()} (i.e., the user has specified two program arguments that are incompatible, one making the other useless). \cppinline{note()} is there to bring complementary information to a previous \cppinline{error()} or \cppinline{warning()}, such as suggestions on how to fix the problem or more detail on where/why the error occurred. Alternatively, it can also be used to inform the user of the progress of a program (in this case, a good practice is to only enable these messages if the user asks for a ``verbose'' output).
\begin{cppcode}
bool verbose = false; // let the user enable this only if needed

std::string datafile = "toto.txt";
if (!file::exist(datafile)) {
    error("cannot open '", datafile, "'");
    note("make sure the program is run inside the data directory");
    return 1; // typically, exit the program
}

if (verbose) {
    note("analysing the data, please wait...");
}

// Do the stuff...
\end{cppcode}

The above code, if the file does not exist, will display:
\begin{bashcode}
error: cannot open 'toto.txt'
note: make sure the program is run inside the data directory
\end{bashcode}

A few words about formatting. First, since the text is meant to be sent to a simple terminal window, there are very few options to affect the look and feel of the output. Each character will be printed as it is. There is no option for boldface or color (that may come in the future though, but it will always be fairly limited). The only thing you can actually control is when to start a new line. This is done with the special character \cppinline{'\n'}, and such a line break is always made at the end of each printed message. If you need a finer control, you will have to come back to the standard C++ functions for output (i.e., \cppinline{std::cout} and the likes).

Second, if you are used to C functions like \cppinline{prinft()}, note from the examples above how they behave quite differently! The \phypp functions do not work with format strings, hence the character \cppinline{'%'} is not special and can be used safely without escaping. One can print a value that is not a string (see the ``Advanced'' section below for more details) simply by adding it to the list of the function arguments, which will result in automatic conversion to string. However, the way this conversion is made is not customizable. Integers will always be displayed in base $10$ (i.e., no hexadecimal or binary format), and floating point numbers will always have a maximum number of digits that depends on the numerical precision. If a fancier output is desired, the conversion to string can always be done manually and the resulting string can then be fed to the print function.

\begin{advanced}
Non-string arguments are internally converted to strings using the \cppinline{std::ostream} \cppinline{operator<<}. This means that all the standard literal types and \phypp vectors can be printed, and that most other types from external C++ libraries will be printable out of the box as well. If you encounter some errors while printing a particular type, this probably means that the \cppinline{operator<<} is missing and you have to write it yourself.

\begin{example}
\begin{cppcode}
// We want to make this structure printable
struct test {
    std::string name;
    int i, j;
};

// We just need to write this function
std::ostream& operator<< (std::ostream& o, const test& t) {
    o << t.name << ":{i=" << i << " j=" << j << "}";
    return o; // do not forget to always return the std::ostream!
}

// The idea is always to rely on the existence of an operator<<
// function for the types that are contained by your structure
// or class. In our case, std::string and int are already printable.
// This is the standard C++ way of printing stuff, but it can be
// annoying to use regularly because the "<<" are taking a lot of
// screen space.

// Now we can print!
test t = {"toto", 5, 12};
print("the value is: ", t);
// ...prints:
// the value is: toto:{i=5 j=12}
\end{cppcode}
\end{example}
\end{advanced}

\funcitem \cppinline|bool prompt(string msg, T& v, string err = "")| \itt{prompt}

This function interacts with the user of the program through the standard output and input (i.e., usually the terminal). It first prints \cppinline{msg}, waits for the user to enter a value and press the Enter key, then try to load it inside \cppinline{v}. If the value entered by the user is invalid and cannot be converted into the type \cppinline{T}, the program asks again and optionally writes an error message \cppinline{err} to clarify the situation.

Currently, the function can only return after successfully reading a value, and always returns \cppinline{true}. In the future, it may fail and return \cppinline{false}, for example after the user has failed a given number of times. If possible, try to keep the possibility of failure into account.

\begin{example}
Consider the following program.
\begin{cppcode}
uint_t age;
if (prompt("please enter your age: ", age,
    "it better just be an integral number...")) {
    print("your age is: ", age);
} else {
    print("you will do better next time");
}
\end{cppcode}

Here is a possible interaction scenario with a naive user:
\begin{bashcode}
please enter your age: 15.5
error: it better just be an integral number...
please enter your age: what?
error: it better just be an integral number...
please enter your age: oh I see, it is 15
error: it better just be an integral number...
please enter your age: ok...
error: it better just be an integral number...
please enter your age: 15
your age is: 15
\end{bashcode}
\end{example}
