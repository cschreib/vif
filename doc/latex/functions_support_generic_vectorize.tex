\funcitem \cppinline|auto vectorize_lambda(T func)| \itt{vectorize_lambda}

This function takes a single argument that must be a C++ lambda. In turn, this lambda can take an arbitrary number of arguments, with the only constraint that the \emph{first} argument must be a scalar, i.e., not a \phypp vector.

The function then creates a new lambda that can be called both with a scalar or a vector as the first argument. This allows easy conversion from legacy functions (e.g., from the standard library) into vectorized functions, and automatically takes care of possible optimizations.

\begin{example}
\begin{cppcode}
// Create a simple lambda
auto lambda = [](double f1, double f2) {
    return f1*f2;
};

// It can be called like this:
double res = lambda(1.5, 2.0);
res; // 3.0

// But if we want to call this function on a vector
// of doubles, we cannot:
vec1d x = {1,2,3,4,5};
vec1d vres = lambda(x, 2.0); // error!

// We have to explicitly adapt this function for
// vectorization. Instead of doing this manually,
// we use vectorize_lambda
auto vlambda = vectorize_lambda([](double f1, double f2) {
    return f1*f2;
});

vec1d vres = vlambda(x, 2.0); // ok!
vres; // {2.0, 4.0, 6.0, 8.0, 10.0}

// We can also use this new lambda with scalar arguments
res = vlambda(1.5, 2.0);
res; // 3.0
\end{cppcode}
\end{example}
