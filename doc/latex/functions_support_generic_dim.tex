\funcitem \cppinline|vec<1,T> flatten(vec<D,T>)| \itt{flatten}

This function transforms a multidimensional vector into a monodimensional vector. The content in memory is exactly the same, so the operation is fast. In particular, if the argument of this function is a temporary, this function is extremely cheap as it produces no copy. The \cppinline{reform} function does the inverse job, and more.

\begin{example}
\begin{cppcode}
vec2i v = {{1,2,3}, {4,5,6}};
vec1i w = flatten(v); // {1,2,3,4,5,6}
\end{cppcode}
\end{example}

\funcitem \cppinline|vec<D1,T> reform(vec<D2,T>, ...)| \itt{reform}

This function transforms a vector into another vector just by changing its dimensions. The content in memory is exactly the same, so the operation is fast. In particular, if the argument of this function is a temporary, this function is extremely cheap as it produces no copy. However, the new dimensions have to generate the same number of elements as the provided vector contains. The \cppinline{flatten} function is a special case of \cppinline{reform}.

\begin{example}
\begin{cppcode}
vec1i v = {1,2,3,4,5,6};
vec2i w = reform(v, 2, 3); // {{1,2,3}, {4,5,6}}
\end{cppcode}
\end{example}

\funcitem \cppinline|vec<N,T>   replicate(T, ...)| \itt{replicate}

      \cppinline|vec<D+N,T> replicate(vec<D,T>, ...)|

This function will take the provided scalar (first version) or vector (second version), and replicate it multiple times according to the provided additional parameters, to generate additional dimensions.

\begin{example}
\begin{cppcode}
// First version
vec1i v = replicate(2, 5);
v; // {2,2,2,2,2} 5 times 2
vec2i w = replicate(2, 3, 2);
w; // {{2,2},{2,2},{2,2}} 3 x 2 times 2

// Second version
vec2i z = replicate(vec1i{1,2}, 3);
z; // {{1,2},{1,2},{1,2}} 3 times {1,2}
// Note that it is not possible to just use a plain initializer list
vec2i z = replicate({1,2}, 3); // error, unfortunately
\end{cppcode}
\end{example}
\end{itemize}

\subsection{Adding/removing elements}

\begin{itemize}
\funcitem \cppinline|vec<1,T> remove(vec<1,T> v, vec1u id)| \itt{remove}

\cppinline|void inplace_remove(vec<1,T>& v, vec1u id)| \itt{inplace_remove}

These functions will remove the elements in \cppinline{v} that have their indices in \cppinline{id}. The only difference between the first and the second version is that the former will first make a copy of the provided vector, remove elements inside this copy, and return it. The second version modifies directly the provided vector, and is therefore faster.

\begin{example}
\begin{cppcode}
// First version
vec1i v = {4,5,2,8,1};
vec1i w = remove(v, {1,3}); // {4,2,1}

// Second version
inplace_remove(v, {1,3});
v; // {4,2,1}
\end{cppcode}
\end{example}

\funcitem \cppinline|void append<N>(vec<D,T1>& t1, vec<D,T2> t2)| \itt{append}

\cppinline|void append(vec<1,T1>& t1, vec<1,T2> t2)|

\cppinline|void prepend<N>(vec<D,T1>& t1, vec<D,T2> t2)| \itt{prepend}

\cppinline|void prepend(vec<1,T1>& t1, vec<1,T2> t2)|

These functions behave similarly to \cppinline{vec::push_back}, in that they will add new elements at the end (\cppinline{append}), but also at the beginning (\cppinline{prepend}) of the provided vector. However, when \cppinline{vec::push_back} can only add new elements from a vector that is one dimension less than the original vector (or a scalar, for monodimensional vectors), these functions will add new elements from a vector of the \emph{same} dimension. These functions are also more powerful than \cppinline{vec::push_back}, because they allow you to choose along which dimension the new elements will be added, through the template parameter \cppinline{N} (note that this parameter is useless and therefore does not exist for monodimensional vectors). The other dimensions must be identical. They are clearly dedicated to a more advanced usage.

\begin{example}
\begin{cppcode}
// For monodimensional vectors
vec1i v = {1,2,3};
vec1i w = {4,5,6};
append(v, w);
v; // {1,2,3,4,5,6}
prepend(v, w);
v; // {4,5,6,1,2,3,4,5,6}

// For multidimensional vectors
vec2i x = {{1,2}, {3,4}};         // x is (2x2)
vec2i y = {{0}, {0}};             // y is (2x1)
vec2i z = {{5,6,7}};              // z is (1x3)
append<1>(x, y);
x; // {{1,2,0}, {3,4,0}}          // x is (2x3)
prepend<0>(x, z);
x; // {{5,6,7}, {1,2,0}, {3,4,0}} // x is (3x3)
\end{cppcode}
\end{example}
